\documentclass{article}
\usepackage[utf8]{inputenc}
\usepackage[T1]{fontenc}
\usepackage{polski}
\usepackage{fancyhdr}
\usepackage{indentfirst}
\usepackage{lastpage}
\usepackage{setspace}
\usepackage{longtable}

\setlength{\parskip}{1ex plus 0.5ex minus 0.2ex}

\pagestyle{fancy}
\title{Specyfikacja implementacyjna projektu indywidualnego \textit{Bieszczadzki Komiwojażer}}

\begin{document}
\begin{titlepage}
\makeatletter
\noindent
\vspace{25pt}
\begin{center}
\LARGE \textsc{\@title}
\end{center}
\makeatother
\vspace{300pt}
\begin{flushright}
\noindent Wykonał: Piotr Ferdynus\\
Sprawdził: mgr inż. Paweł Zawadzki\\
Data: 11.11.2019\\
\end{flushright}


\thispagestyle{empty}
\end{titlepage}

\rhead{Piotr Ferdynus 299244}
\lhead{}
\cfoot{\thepage \hspace{1pt} / \pageref{LastPage}}
\setcounter{page}{2}

\section{Wprowadzenie}

\subsection{Cel dokumentu}
Sprecyzowanie sposobu implementacji projektu, który szczegółowo opisany został w specyfikacji funkcjonalnej. Określenie logiki działania programu, opisanie użytych struktur danych i zastosowanych algorytmów.

\subsection{Cel projektu}

\subsection{Użytkownik końcowy}

\section{Środowisko deweloperskie}
Opis parametrów, środowisk i oprogramowania, które zostanie użyte podczas pracy nad projektem.

\subsection{Parametry sprzętowe}
Podczas procesu wytwarzania oprogramowania zostaną wykorzystane dwie stacje robocze o następującej specyfikacji:

\begin{verbatim}
    Procesor AMD Ryzen 5 2500U
    Zintegrowana karta graficzna Radeon Vega 8 Mobile
    Pamięć RAM DDR4 8GB
    Windows 10 Home wersja 1903
    
    Procesor Intel Core i5 8th gen
    Karta graficzna NVidia Geforce GTX 1060 4GB
    Pamięć RAM DDR4 8GB
\end{verbatim}

\subsection{Oprogramowania}

Na obu komputerach zostało zainstalowane oprogramowanie pozwalające na pracę w języku programowania Java:

\begin{verbatim}
    SDK Java 11.0.2 2019-01-15 LTS
    Java(TM) SE Runtime Environment 18.9
    Java HotSpot(TM) 64-Bit Server VM 18.9
    IDE Intellij IDEA Ultimate 2019.1.1 
\end{verbatim}

\section{Zasady wersjonowania}

Podczas realizacji projektu zostały przyjęte ogólnie akceptowane zasady wersjonowania projektów informatycznych. Numer wersji występuje w postaci \textit{X.Y.Z}, gdzie X, Y i Z reprezentują liczby naturalne. Człon \textit{X}, indeksowany od zera, to iteracja wydań niekompatybilnych wstecznie lub wnoszących istotne zmiany w funkcjonalności oprogramowania. Człon \textit{Y} indeksowany od jedynki, reprezentuje mniejszy przyrost funkcjonalności aplikacji. Ostatnia część \textit{Z}, rozpoczynająca się od zera, informuje o poprawie błędów i niedociągnięć wykrytych przy wcześniejszych wersjach programu.

\section{Diagram klas}


\section{Struktury danych}


\section{Algorytmy}


\end{document}