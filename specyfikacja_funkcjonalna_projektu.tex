\documentclass{article}
\usepackage[utf8]{inputenc}
\usepackage[T1]{fontenc}

\title{Specyfikacja funkcjonalna projektu bieszczadzki komiwojażer  - wersja alfa}
\author{Piotr Ferdynus}
\date{04.11.2019}
\begin{document}
\maketitle

\part{Alpha version}
\section{Wprowadzenie}
\subsection{Cel dokumentu}
Określenie pełnej funkcjonalności projektu bieszczadzki komiwojażer , jak również sprecyzowanie sposobu interakcji z programem, uzyskiwanego wyniku, obsługi błędów oraz przebiegu testowania gotowego oprogramowania.
\subsection{Cel projektu}
Umożliwienie łatwego i szybkiego wyszukiwania optymalnej trasy pomiędzy wybranymi przez użytkowanika punktami, kierując się długością trwania trasy jak również wysokością koniecznych do poniesienia kosztów, poprzez opracowanie i implementację odpowiedniego algorytmu.
\subsection{Użytkownik końcowy}
Projekt w założeniu jest przeznaczony dla sprecyzowanej grupy odbiorców -- studentów Joachima, Alojzego, Cypriana i Eustachego, jednak ze względu na jego dużą uniwersalność może być zastosowany przez każdą osobę lub grupę zorganizowaną pragnącą znaleźć najbardziej optymalny wariant trasy w Bieszczadach.
\section{Uruchomienie programu}
Uruchomienie programy dokonuje się z poziomu wiersza poleceń, jako argumenty podając ścieżkę do pliku konfiguracyjnego, miejsce startu oraz, ewentualnie, plik zawierający miejsca podróży, których odwiedzenie jest warunkiem koniecznym.

tutaj dwa przykłayd wywołania

\section{Dane wejściowe}
Dane wejściowe wprowadzane do programu:
\subsection{Plik konfiguracyjny}
Plik zawierający dane o wszystkich możliwych punktach podróży oraz o czasach przejścia pomiędzy tymi punktami. Struktura pliku kofiguracyjnego:

\begin{verbatim}
### Miejsca podróży
Lp. | ID_miejsca | Nazwa miejsca początkowego | Opis miejsca |
1. | A | Koliba Studencka Politechniki Warszawskiej | Baza wypadowa (miejsce rozpoczęcia wędrówki) |
2. | B | Jawornik | Szczyt (1021) |
3. | C | Rabia Skała | Szczyt (1199) |
4. | D | Dziurkowiec | Szczyt (1189) |
5. | E | Okrąglik | Szczyt (1101) |
6. | F | Fereczata | Szczyt (1102) |

### Czas przejścia
Lp. | ID_miejsca_początkowego (S) | ID_miejsca_końcowego (E) | Czas S -> E | Czas E -> S | Jednorazowa opłata za przejście trasą (zł) |
1. | A | B | 2:00 | 3:00 | -- |
2. | A | C | 3:00 | 3:30 | 5 |
3. | B | C | 1:30 | 1:30 | -- |
4. | B | D | 1:00 | 1:30 | -- |
5. | B | E | 1:00 | 1:30 | -- |
6. | C | D | 3:00 | 2:00 | -- |
7. | D | F | 4:00 | 3:00 | -- |
8. | E | F | 0:30 | 0:30 | -- |
\end{verbatim}
\subsection{Miejsce startu}
Argument zawierający wartość ID\_miejsca, z którego ma rozpocząć i w którym ma zakończyć się planowana wędrówka.
\subsection{Wybrane miejsca podróży}
Plik zawierający zbiór wartości ID\_miejsca wszystkich miejsc, które są koniecznymi punktami wybranej wędrówki. Ten argument wywołania jest opcjonalny, w przypadku jego braku program określi optymalną trasę przejścia przez wszystkie miejsca znajdujące się w pliku konfiguracyjnym.
\section{Dane wyjściowe}
Dane wyjściowe zostaną zapisane w postaci czterech wersów danych: pierwszy wers zawiera trasę obliczoną jako rezultat działania programu w postaci ciągu wartości ID\_miejsca rozdzielonych symbolem strzałki " -> ". W drugim wersie znajduje się rozszerzenie informacji zawartych powyżej poprzez zastąpienie wartości ID\_miejsca nazwą miejsca. Następnie zostaje podany przewidywany czas wędrówki w godzinach i minutach, oraz sumaryczny koszt wstępu na szlak. Całość zostaje zapisana w pliku wynikowym w formacie \textit{.txt} o nazwie "nazwa pliku" w folderze z poziomu którego wywołany został program. 
(dodaj możliwy wygląd pliku wynikowego)
\section{Scenariusz uruchomienia}
Program po uruchomieniu wyświetli komunikat o rozpoczęciu pracy. Następnie rozpocznie interpretację argumentów wejściowych, wyświetlając na bieżąco komunikat o ewentualnych błędach wraz z ich dokładnym opisem, oraz informacją, czy błąd okazał się krytyczny i uniemożliwił kontynuowanie pracy. Po wczytaniu odpowiednich plików i argumentów wejściowych program wyświetli komunikat o poprawnym bądź niepoprawnym zakończeniu odczytu danych wejściowych. Następnie wykona zlecone mu zadanie, a fakt pomyślnego zakończenia działania zaanonsuje odpowiednim komunikatem.
\section{Opis sytuacji wyjątkowych}
\begin{tabular}{|c|c|c|} \hline
& & \\
opis sytuacji wyjątkowej & sposób obsługi sytuacji & czy kończy pracę programu \\ 
& & \\ \hline
błąd pliku wejściowego & wyświetlenie komunikatu & tak\\ \hline
\end{tabular}
\section{Testowanie}
Testowaniu będą podlegać następujące przypadki:
\begin{enumerate}
    \item Wywołanie programu bez argumentów wejściowych
    \item Podanie nieprawidłowej nazwy pliku
\end{enumerate}


\end{document}
