\documentclass{article}
\usepackage[utf8]{inputenc}
\usepackage[T1]{fontenc}
\usepackage{polski}
\usepackage{fancyhdr}
\usepackage{indentfirst}
\usepackage{lastpage}
\usepackage{setspace}
\usepackage{longtable}

\setlength{\parskip}{1ex plus 0.5ex minus 0.2ex}

\pagestyle{fancy}
\title{Specyfikacja funkcjonalna projektu indywidualnego \textit{Bieszczadzki Komiwojażer}}

\begin{document}
\begin{titlepage}
\makeatletter
\noindent
\vspace{25pt}
\begin{center}
\LARGE \textsc{\@title}
\end{center}
\makeatother
\vspace{300pt}
\begin{flushright}
\noindent Wykonał: Piotr Ferdynus\\
Sprawdził: mgr inż. Paweł Zawadzki\\
Data: 06.11.2019\\
\end{flushright}


\thispagestyle{empty}
\end{titlepage}

\rhead{Piotr Ferdynus 299244}
\lhead{}
\cfoot{\thepage \hspace{1pt} / \pageref{LastPage}}
\setcounter{page}{2}

\section{Wprowadzenie}
Wstęp do tematyki projektu i opis rozwiązywanego problemu.
\subsection{Cel dokumentu}
Określenie pełnej funkcjonalności projektu \textit{Bieszczadzki Komiwojażer}, jak również sprecyzowanie sposobu interakcji z programem, formy uzyskiwanego rezultatu, sposobu obsługi sytuacji wyjątkowych oraz przebiegu testowania gotowego oprogramowania.

\subsection{Cel projektu}
Umożliwienie bezproblemowego i szybkiego znalezienia optymalnego wariantu trasy pomiędzy wybranymi przez użytkownika punktami, kierując się czasem potrzebnym na pokonanie trasy oraz, w drugiej kolejności, wysokością koniecznych do poniesienia kosztów.

\subsection{Użytkownik końcowy}
Program przeznaczony jest dla osób o ograniczonym czasie i mocno ograniczonych nakładach finansowych, które chciałyby doświadczyć wędrówki po malowniczej krainie Bieszczad, odwiedzając wszystkie wybrane przez siebie miejsca w jak najkrótszym czasie.

\vspace{15pt}
\section{Uruchomienie programu}
Uruchomienie programu dokonuje się z poziomu wiersza poleceń. Do prawidłowego działania wymagany jest plik konfiguracyjny oraz indeks miejsca startu, opcjonalną daną jest zbiór wybranych przez użytkownika miejsc, przez które musi przebiegać trasa. Pliki zostają przekazane poprzez podanie ścieżki do pliku jako argument wywołania. Szczegółowe informacje na temat danych wejściowych znajdują się w sekcji \textit{Dane wejściowe}. 

\vspace{10pt}
Przykładowy sposób wywołania programu:
\vspace{-8pt}
\begin{verbatim}
    .\pathfinder data\plik_konfiguracyjny A
    .\pathfinder data\plik_konfiguracyjny B \data\wybrane_miejsca
\end{verbatim}

\vspace{15pt}
\section{Dane wejściowe}
Szczegółowy opis danych wejściowych dostarczanych do programu.

\subsection{Pliki}
Pliki są przekazywane poprzez podanie ścieżki do pliku jako argumentu wywołania. Kolejne kolumny są rozdzielone znakiem '|', a kolejne wiersze znakiem nowej linii. Poszczególne części mogą zawierać nagłówki, ale nie jest to wymagane. Liczby porządkowe poszczególnych części pliku muszą być posortowane rosnąco i nie mogą się powtarzać. Linijka rozpoczynająca się symbolem '\#' jest ignorowana przez program.

\subsubsection{Plik konfiguracyjny}
Plik zawierający dane niezbędne do prawidłowego działania programu: dane o wszystkich możliwych punktach podróży oraz o czasie przejścia pomiędzy bezpośrednio połączonymi punktami. Struktura pliku konfiguracyjnego składa się z dwóch części -- pierwsza część zawiera informacje o miejscach podróży w~następującej postaci:

\vspace{5pt}
\begin{verbatim}
    Lp. | ID_miejsca | Nazwa miejsca | Opis miejsca
\end{verbatim}
\vspace{5pt}

Kolejne kolumny oznaczają odpowiednio -- liczbę porządkową, \textbf{unikalny} indeks miejsca, nazwę miejsca i opis słowny miejsca. Druga część pliku zawiera informacje o czasie przejścia pomiędzy połączonymi punktami w następującej strukturze:

\vspace{5pt}
\begin{verbatim}
    Lp. | ID_S | ID_E | S -> E | E -> S | opłata 
\end{verbatim}
\noindent Kolumny oznaczają odpowiednio -- liczbę porządkową, indeks miejsca początkowego zawarty w pierwszej części pliku, indeks miejsca końcowego zawarty w pierwszej części pliku, czas potrzebny na pokonanie trasy z miejsca początkowego do miejsca końcowego, czas potrzebny na pokonanie trasy z miejsca końcowego do miejsca początkowego i jednorazową opłatę za wstęp wyrażoną w~złotówkach.

\vspace{10pt}
Przykład pliku konfiguracyjnego:
\vspace{-8pt}
\begin{verbatim}
    ### Miejsca podróży
    1. | B | Jawornik | Szczyt (1021) |
    2. | C | Rabia Skała | Szczyt (1199) |
    3. | D | Dziurkowiec | Szczyt (1189) |
    ### Czas przejścia
    1. | A | B | 2:00 | 3:00 | -- |
    2. | A | C | 3:00 | 3:30 | 5 |
    3. | B | C | 1:30 | 1:30 | -- |
    4. | B | D | 1:00 | 1:30 | -- |
    5. | B | E | 1:00 | 1:30 | -- |
    6. | C | D | 3:00 | 2:00 | -- |
\end{verbatim}

\subsubsection{Wybrane miejsca podróży}
Plik zawierający wartości ID\_miejsca wszystkich miejsc, które są koniecznymi punktami planowanej wędrówki. Ten argument wywołania jest opcjonalny, w~przypadku jego braku program określi optymalną trasę przejścia przez wszystkie miejsca znajdujące się w pliku konfiguracyjnym. Struktura pliku przedstawia się następująco: 
\begin{verbatim}
    Lp. | ID_miejsca
\end{verbatim}
\noindent Kolumny reprezentują kolejno: liczbę porządkową i ID\_miejsca, określone w~podanym wcześniej pliku konfiguracyjnym. 

\vspace{10pt}
Przykładowy plik wybranych miejsc:
\vspace{-8pt}
\begin{verbatim}
    ### Wybrane miejsca podróży
    Lp. | ID_miejsca |
    1. | A |
    2. | B |
    3. | E |
\end{verbatim}


\subsection{Miejsce startu}
Argument stanowiący wartość ID\_miejsca, w którym znajduje się planowane miejsce startu i zakończenia wędrówki.



\newpage
\section{Dane wyjściowe}
Optymalna trasa marszu, uzyskana jako rezultat pracy programu zostaje zapisana do pliku w formacie \textit{.txt} o nazwie \textit{"optymalna\_trasa"} w folderze, z~poziomu którego wywołany został program. Dane wyjściowe zostają sformatowane w następujący sposób: pierwszy wers zawiera trasę w~postaci ciągu wartości ID\_miejsca rozdzielonych symbolem strzałki " -> ". W kolejnych wersach znajduje się rozszerzenie tej informacji poprzez zastąpienie wartości ID\_miejsca jego nazwą. Następnie zostaje podany przewidywany czas wędrówki wyrażony w godzinach i minutach oraz sumaryczny koszt wstępu na szlak.

\vspace{10pt}
Przykładowy plik wynikowy programu:
\vspace{-8pt}
\begin{verbatim}
    F -> G -> H -> A -> B -> D -> F
    Ustrzyki Górne 
    -> Połonina Caryńska 
    -> Bacówka Pod Małą Rawką 
    -> Mała Rawka 
    -> Wielka Rawka 
    -> Ustrzyki Górne
    Czas: 6 godzin 21 minut
    Koszt: 5 zł
\end{verbatim}

\vspace{15pt}
\section{Scenariusz uruchomienia}
Program po uruchomieniu wyświetla komunikat o rozpoczęciu pracy. Następnie rozpocznie interpretację argumentów wejściowych, wyświetlając na bieżąco komunikat o ewentualnych błędach wraz z ich dokładnym opisem oraz informacją, czy błąd okazał się krytyczny i uniemożliwił kontynuowanie pracy. Po wczytaniu odpowiednich plików i argumentów wejściowych program wyświetli komunikat o poprawnym bądź niepoprawnym zakończeniu odczytu danych wejściowych. Następnie program przystąpi do wyznaczania optymalnej trasy, a fakt pomyślnego zakończenia działania zasygnalizuje odpowiednim komunikatem.

\newpage
\section{Opis sytuacji wyjątkowych}
Opis obsługi błędów i komunikacji z użytkownikiem po wystąpieniu niestandardowych sytuacji, które mogą wydarzyć się podczas pracy programu.
\vspace{10pt}

\begin{sloppypar}
\begin{spacing}{1.2}
\begin{longtable}{| p{0.35\linewidth} | p{0.65\linewidth}|} \hline
\textbf{opis sytuacji}   &   \textbf{sposób obsługi} \\ \hline \hline
brak podanych argumentów & Przerwanie działania programu. Wyświetlenie pomocy dotyczącej uruchomienia programu. \\ \hline
brak pliku wejściowego pod wskazanym adresem &   Przerwanie działania programu. Wyświetlenie komunikatu: "[podana ścieżka do pliku] -- plik nie istnieje". \\ \hline
brak indeksu miejsca rozpoczęcia & Przerwanie działania programu. Wyświetlenie komunikatu: "brak podanego miejsca startu".\\ \hline
pusty plik  & Przerwanie działania programu. Wyświetlenie komunikatu: "podany plik [podana ścieżka do pliku] jest pusty".  \\ \hline
powtórzony indeks miejsca & Pominięcie powtórzonego indeksu. Wyświetlenie komunikatu: "wers: [numer wersu] | indeks [indeks] występuje już w tym pliku w wersie [numer powtórzonego wersu]; pomijam". \\ \hline
podane miejsca nie są ze sobą połączone &  Przerwanie działania programu. Wyświetlenie komunikatu: "brak trasy; punkty [id\_punktu] i~[id\_punktu] nie są ze sobą połączone".\\ \hline
podany czas trasy jest ujemny & Przerwanie działania programu. Wyświetlenie komunikatu: "wers: [numer wersu] | czas trasy nie może być ujemny.\\ \hline
podany koszt wstepu na trasę jest ujemny & Przyjęcie kosztu trasy 0 zł. Wyświetlenie komunikatu: "wers: [numer wersu] | opłata wstępu nie może być ujemna; przyjmuję wstęp darmowy".\\ \hline
miejsce startu nie znajduje się w pliku konfiguracyjnym &  Przerwanie działania programu. Wyświetlenie komunikatu "brak miejsca o ID\_miejsca [id\_miejsca]". \\ \hline
nieprawidłowe id\_miejsca w schemacie tras &  Przerwanie działania programu. Wyświetlenie komunikatu: "wers: [numer wersu] | brak miejsca o~ID\_miejsca [id\_miejsca]".\\ \hline
\end{longtable}
\end{spacing}
\end{sloppypar}


\section{Testowanie}
Opis sposobu testowania funkcjonalności programu.

\subsection{Sytuacje wyjątkowe}
Sprawdzenie poprawności obsługi sytuacji niestandardowych.

\begin{enumerate}
        \item Wywołanie programu bez argumentów wejściowych.
        \item Podanie nieprawidłowej nazwy pliku.
        \item Podanie ścieżki do pustego pliku.
        \item Plik konfiguracyjny nie zawiera schematu tras.
        \item Plik zawiera powtórzone indeksy miejsca.
        \item Plik zawiera ujemne czasy tras i koszty wstępu.
\end{enumerate}

\subsection{Standardowe działanie programu}
Sprawdzenie prawidłowości algorytmu przy poprawnych danych wejściowych.
    \begin{enumerate}
        \item Wywołanie programu bez podania wybranych miejsc.
        \item Wywołanie programu z podaniem wybranych miejsc.
        \item Wywołanie programu dla dużej ilości danych.
    \end{enumerate}



\end{document}